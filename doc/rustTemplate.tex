\IfFileExists{t2doc.cls}{
    \documentclass[documentation]{subfiles}
}{
    \errmessage{Error: could not find 't2doc.cls'}
}

\begin{document}

\trantitle
    {rustTemplate} % Plugin name
    {rustTemplate} % Short description
    {Tranalyzer Development Team} % author(s)

\section{rustTemplate}\label{s:rustTemplate}

\subsection{Description}
The rustTemplate plugin analyzes ...

\subsection{Dependencies}

%\traninput{file} % use this command to input files
%\traninclude{file} % use this command to include files

%\tranimg{image} % use this command to include an image (must be located in a subfolder ./img/)

\subsubsection{External Libraries}
This plugin depends on the {\bf XXX} library.
\paragraph{Ubuntu:} {\tt sudo apt-get install XXX}
\paragraph{Arch:} {\tt sudo pacman -S XXX}

\subsubsection{Other Plugins}
This plugin requires the {\bf XXX} plugin.

\subsubsection{Required Files}
The file {\tt file.txt} is required.

\subsection{Configuration Flags}
The following flags can be used to control the output of the plugin:
\begin{longtable}{lcll}
    {\bf Name} & {\bf Default} & {\bf Description} & {\bf Flags}\\
    \hline\endhead
    {\tt FLAG1} & 0 & Whether (1) or not (0) to activate FLAG1\\
    {\tt OPT2} & 1 & 0: no OPT2, 1: one OPT2, 2: two OPT2 & {\tt FLAG1=1}\\
\end{longtable}

\subsection{Flow File Output}
The rustTemplate plugin outputs the following columns:
\begin{longtable}{llll}
    {\bf Column} & {\bf Type} & {\bf Description} & {\bf Flags}\\
    \hline\endhead
    {\tt \nameref{rustTemplateStat}} & H8 & Status & \\
    {\tt \hyperref[rustTemplateStat]{col1}} & S & describe col1 (string) & {\tt NUDEL=1} (only shown if NUDEL=1)\\
    {\tt col2} & U32 & describe col2 (uint32) & \\
    {\tt col3} & I16 & describe col3 (int16) & \\
    {\tt col4} & H8 & describe col4 (hex/uint8) & \\
    {\tt col5} & F\_D & describe col5 (compound (float, double)) & \\
\end{longtable}

\subsubsection{rustTemplateStat}\label{rustTemplateStat}
The {\tt rustTemplateStat} column is to be interpreted as follows:
\begin{longtable}{rl}
    {\bf rustTemplateStat} & {\bf Description}\\
    \hline\endhead
    {\tt 0x01} & Flow is rustTemplate\\
\end{longtable}

\subsection{Packet File Output}
In packet mode ({\tt --s} option), the rustTemplate plugin outputs the following columns:
\begin{longtable}{llll}
    {\bf Column} & {\bf Type} & {\bf Description} & {\bf Flags}\\
    \hline\endhead
    {\tt rustTemplateCol1} & I8 & describe col1 & \\
\end{longtable}
 
\subsection{Plugin Report Output}
The following information is reported:
\begin{itemize}
    \item Number of XXX packets
    \item Aggregated status flags ({\tt\nameref{rustTemplateStat}})
\end{itemize}

\subsection{Additional Output}
Non-standard output:
\begin{itemize}
    \item {\tt PREFIX\_suffix.txt}: description
\end{itemize}

\subsection{Post-Processing}

\subsection{Example Output}

\subsection{Known Bugs and Limitations}

\subsection{TODO}
\begin{itemize}
    \item TODO1
    \item TODO2
\end{itemize}

\subsection{References}
\begin{itemize}
    \item \href{https://tools.ietf.org/html/rfcXXXX}{RFCXXXX}: Title
    \item \url{https://www.iana.org/assignments/}
\end{itemize}

\end{document}
